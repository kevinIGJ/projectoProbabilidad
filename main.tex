\documentclass[12pt,letterpaper]{article}
\usepackage[utf8]{inputenc}
\usepackage[spanish]{babel}
\spanishdecimal{.}
\usepackage{graphicx}
\usepackage{enumerate}
\usepackage[margin=0.50in]{geometry}
\usepackage[nottoc]{tocbibind}
\setlength{\parindent}{0cm}
\begin{document}

\thispagestyle{empty}
\begin{center}
\includegraphics[width=0.6\textwidth]{logo.png}\\
\vspace{1.5cm}
\Large \sc  Tecnológico de Monterrey\\
	            Campus Hidalgo


\vspace{1.5cm}
\huge \bf
\end{center}
\begin{flushright}
\bf \huge
\emph{Proyecto final
}
\end{flushright}
\vspace{1.5cm}
\rightline{\Large \bf Erhart Fabian Castillo Castellanos A01274089}
\vspace{0.2cm}
\rightline{\Large \bf Kevin Israel Guzmán Jiménez A01275139}
\vspace{0.2cm}
\rightline{\Large \bf Fernando Garrido Del Valle A01273241}
\vspace{0.2cm}
\rightline{\Large \bf José Angel Olvera López A01275136}
\vspace{1.5cm}
\rightline{Probabilidad y estadística}
\rightline{Ronald Richard Jiménez}
\vspace{6.0 cm}

\begin{center}
	\large \bf 6 de mayo del 2019
\end{center}


\tableofcontents
\newpage
\section{Primer ejercicio}
Un fabricante de automóviles se preocupa por
una falla en el mecanismo de freno de un modelo específico. En raras ocasiones la falla puede causar una
catástrofe al manejarlo a alta velocidad. La distribución
del número de automóviles por año que experimentará
la catástrofe es una variable aleatoria de Poisson con $\lambda$= 5. \par
a) ¿Cuál es la probabilidad de que, a lo sumo, 3 automóviles
por año de ese modelo específico sufran
una catástrofe?\par
b) ¿Cuál es la probabilidad de que más de un automóvil por año experimente una catástrofe?\cite{walpole}
\subsection{Antecedentes}
La función de densidad de probabilidad para $x$ es: 
\begin{equation}
    P\left( x \right) = \frac{{e^{ - \lambda } \lambda ^x }}{{x!}}
\end{equation}
\subsection{Procedimiento}
a) Determinaremos la probabilidad de que al menos tres carros experimenten una catástrofe: 
\begin{equation}
    P(X \leq 3)=P(X=0)+P(X=1)+P(X=2)+P(X=3)
\end{equation}
\begin{equation}
    =p(0;5)+p(1;5)+p(2;5)+p(3;5)
\end{equation}
\begin{equation}
    =\frac{e^{-5}(5)^{0}}{0!}+\frac{e^{-5}(5)^{0}}{1!}+\frac{e^{-5}(5)^{0}}{2!}+\frac{e^{-5}(5)^{0}}{3!}
\end{equation}
\begin{equation}
=0.006738+0.03369+0.084224+0.140374
\end{equation}
\begin{equation}
\textbf{=0.2650}
\end{equation}\newline
b) Ya que no se puede calcular probabilidades $P(X\geq x)$, solamente $P(X\leq x)$ y asumiendo que la probabilidad total del evento es $1$ entonces:
\begin{equation}
    P(X>1)=1-P(X \leq 1)
\end{equation}
\begin{equation}
    =1-P(X=0)-P(X=1)
\end{equation}
\begin{equation}
    =1-p(0;5)+p(1;5)
\end{equation}
\begin{equation}
    =1-\frac{e^{-5}(5)^{0}}{0!}-\frac{e^{-5}(5)^{1}}{1!}  
\end{equation}
\begin{equation}
    =1-0.006738-0.03369
\end{equation}
\begin{equation}
    \textbf{=0.9596}
\end{equation}

\section{Segundo ejercicio}
Denote con p(y) la función de probabilidad asociada con una variable aleatoria de Poisson con media $\lambda$.
a) Demuestre que la relación entre probabilidades sucesivas satisface la igualdad $\frac{p(y)}{p(y-1)}=\frac{\lambda}{y}$, \newline para $y$ =
1, 2, . . .\newline
b) ¿Para qué valores de $y$ es $p(y) > p(y - 1)$?\newline
c) Observe que el resultado del inciso a implica que las probabilidades de Poisson aumentan por un tiempo cuando y aumenta y disminuyen de ahí en adelante. Demuestre que $p(y)$ es maximizada cuando $y =$ al máximo entero menor o igual que $\lambda$.\cite{weckerly}
\subsection{Antecedentes}
La función de densidad de probabilidad para $y$ es: 
\begin{equation}
    P\left( X=y \right) =p(y)= \frac{{e^{ - \lambda } \lambda ^y }}{{y!}}
\end{equation}
\subsection{Procedimientos}
a) Sustituyendo $y$ con $y-1$:
\begin{equation}
    P\left( X=y-1 \right) =p(y-1)= \frac{{e^{ - \lambda } \lambda ^{y-1} }}{{(y-1)!}}
\end{equation}
La razón entre las ecuaciones (13) y (14) es:
\begin{equation}
    \frac{p(y)}{p(y-1)}=\frac{ \frac{{e^{ - \lambda } \lambda ^y }}{{y!}}}{\frac{{e^{ - \lambda } \lambda ^{y-1} }}{{(y-1)!}}}
\end{equation}
Simplificando la ecuación (15) se obtiene: 
\begin{equation}
    \frac{p(y)}{p(y-1)}=\frac{\frac{\lambda}{y}}{\frac{1}{1}}= {\bf \frac{\lambda}{y}}
\end{equation}
b) Se puede decir que $p(y)>p(y-1)$ sucede cuando:
\begin{equation}
\frac{p(y)}{p(y-1)}>0
\end{equation}
Usando la ecuación (16):
\begin{equation}
\frac{\lambda}{y}>0
\end{equation}
Al ser $y>0$, se obtiene:
\begin{equation}
\lambda>y
\end{equation}
Entonces los valores de $y$ menores que $\lambda$ satisfacen la desigualdad.\newline \newline
c) Sabemos que $p(y)$ incrementa si $p(y) \geq p(y-1) $ y $p(y-1)<p(y)$, entonces la posición cuando el incremento pasa al decremento se cumple cuando:
\begin{equation}
\frac{p(y)}{p(y-1)} \geq 0 \; \; y \; \; \frac{p(y+1)}{p(y)}<0
\end{equation}
Usando la ecuación (16):
\begin{equation}
\frac{\lambda}{y} \geq 0 \; \; y \; \; \frac{\lambda}{y+1}<0
\end{equation}
Entonces se obtiene: 
\begin{equation}
y \leq \lambda \; \; y \; \; (y+1)>\lambda
\end{equation}
Entonces $y$ es el mayor entero menor o igual que $\lambda$.
\newpage
\section{Tercer ejercicio}
Supóngase que se conoce que normalmente el 10\% de la producción en un proceso de producción está defectuosa. Un inspector de calidad extrae una muestra 
de 50 piezas terminadas y las inspecciona. Este inspector desea saber cual es la 
probabilidad de que en el lote de 50 hayan\\
a) Exactamente tres (3) piezas con defectos. \\
b) A lo más dos (2) piezas defectuosas. \\
c) Más de tres (3) piezas con defectos. \cite{Marcos}
\subsection{Antecedentes}
La función de densidad de probabilidad es: \newline
\begin{equation}
    \ b(x;n,p)=(_{x}^{n}\textrm{})p^{x}(1-p)^{n-x}\
\end{equation}

\subsection{Procedimientos}
En este caso el numero de muestra seria 50
\begin{equation}
    n=50
\end{equation}
Y la probabilidad de éxito
\begin{equation}
    p=0.10
\end{equation}
a) Por lo cual la formula en el inciso ``a'' quedaría así
\begin{equation}
    \ p(x=3)=b(3;50,0.10)=(_{0.10}^{50}\textrm{})p^{3}(1-0.10)^{50-0.10}\
\end{equation}
Por lo tanto 
\begin{equation}
     \ b(3;50,0.10)=0.1385651
\end{equation}
b) En el inciso ``b'' se esta pidiendo de forma acumulativa por lo cual
\begin{equation}
     \ p(x \leq 2)=b(x\leq 2;50,0.10)=p(0)+p(1)+p(2)
\end{equation}
\begin{equation}
     p(0)= 0.005153775
\end{equation}
\begin{equation}
    p(1)= 0.02863208 
\end{equation}
\begin{equation}
     p(2)= 0.0779429
\end{equation}
Por lo tanto
\begin{equation}
    p(x \leq 2)= 0.005153775+0.02863208+0.0779429 
\end{equation}
\begin{equation}
    p(x \leq 2)= 0.1117288
\end{equation}
c) En el inciso ``c'' al pedir x $\geq$ 3 se representa de la siguiente forma
\begin{equation}
     \ p(x>3)=b(x>3;50,0.10)=1-p(x \leq 3)
\end{equation}
Lo que es igual a
\begin{equation}
    \ p(x>3)=1-[p(0)+p(1)+p(2)+p(3)]
\end{equation}
Por lo tanto
\begin{equation}
     \ p(x>3)=1-[0.005153775+0.02863208+0.0779429+0.1385651] 
\end{equation}
\begin{equation}
     \ p(x>3)=1-[0.2502939] 
\end{equation}
\begin{equation}
     \ p(x>3)=0.7497061
\end{equation}




\newpage
\section{Cuarto ejercicio}
Un experto en la materia sabe que el precio de una determinada vivienda está comprendido entre 220 y 232 miles de euros; también sabe que el precio al cual será más probablemente vendida se encuentra en torno a los 224 miles de euros. Suponiendo que la variable aleatoria X, que representa el valor de mercado de la vivienda, tiene la función de densidad:
\begin{equation}
    f(x)=K(x-220)^p(232-x)^q
\end{equation}
se pide:
\begin{itemize}
    \item Encontrar los valores de \textit{p y q} tales que $p+q=3$.
    \item Determinar el valor de \textit{K}.
    \item Calcular el valor esperado de \textit{X}.\cite{ejercicios}
\end{itemize}
\subsection{Procedimiento}
a)
\begin{equation}
    ln f(x)=ln\; K +p\; ln(x-220)+q\; ln(232-x)
\end{equation}
Derivando respecto a x e igualando a 0 se tiene:
\begin{equation}
    \frac{dlnf\;(x)}{dx}=\frac{p}{x-220}-\frac{q}{232-x}
\end{equation}
Por tanto,
\begin{equation}
    \frac{p}{x-220}-\frac{q}{232-x}=0\Leftrightarrow \frac{p}{4}-\frac{q}{8}=0\Leftrightarrow q=2p
\end{equation}
Si ahora se tiene en cuenta que $p+q=3$, podemos determinar los valores p y q resolviendo el sistema:
\[ 
\left \{
  \begin{tabular}{ccc}
  q=2p\\
  p+q=3 
  \end{tabular}
\right \}
\]
Cuya solución es $p=1$ y $q=2$.\\
b) Sustituidos p y q por sus valores calculados en el apartado a) se tiene que:
\begin{equation}
    \int\limits_{220}^{232} f(x)=K \int\limits_{220}^{232}(x-220)(232-x)^2\;dx=1 
\end{equation}
Por tanto:
\begin{equation}
    K=\frac{1}{ \int\limits_{220}^{232}(x-220)(232-x)^2\;dx}
\end{equation}
Por otra parte:
\begin{equation}
    (x-220)(232-x)^2=(x-220)(x^2-464x+53824)\\=x^3-684x^2+155944x-11841280
\end{equation}


Con lo cual:
\begin{equation}
    \int\limits_{220}^{232} (x-220)(232-x)^{2}dx=\int\limits_{220}^{232}(x^{3}-684x^{2}+155904x-11841280)dx 
\end{equation}
\begin{equation}
    =\bigg( \frac{232^{4}}{4}-\frac{684\cdot232^{3}}{3}+\frac{155904\cdot232^{2}}{2}-11841280\cdot232 \bigg)
\end{equation}
\begin{equation}
    -\bigg( \frac{220^{4}}{4}-\frac{684\cdot220^{3}}{3}+\frac{155904\cdot220^{2}}{2}-11841280\cdot220 \bigg)=1728
\end{equation}
Por tanto, $K=\frac{1}{1728}$\\
c) $E(x)=$
\begin{equation}
    \frac{1}{1728}\int\limits_{220}^{232}x(x^{3}-684x^{2}+155904x-11841280)dx 
\end{equation}
\begin{equation}
    =\frac{1}{1728}\bigg( \frac{232^{5}}{5}-\frac{684\cdot232^{4}}{4}+\frac{155904\cdot232^{3}}{3}-\frac{11841280\cdot232^{2}}{2} \bigg)
\end{equation}
\begin{equation}
    -\frac{1}{1728}\bigg( \frac{220^{5}}{5}-\frac{684\cdot220^{4}}{4}+\frac{155904\cdot220^{3}}{3}-\frac{11841280\cdot220^{2}}{2} \bigg)=224.8
\end{equation}


\section{Quinto ejercicio}
Verifique que $Var(x)=\frac{\alpha}{\lambda^{2}};$ Cuando $x$ es una variable aleatoria con distribución gamma con parámetros $\alpha$ y $\lambda$.\cite{ross}
\subsection{Antecedentes}
Función Gamma:
\begin{equation}
    \Gamma \left( \alpha \right) = \int\limits_0^\infty {x^{\alpha - 1} e^{-x} dx}
\end{equation}
Una propiedad de la función Gamma, cuando $\alpha>1$:
\begin{equation}
    \Gamma \left( \alpha \right) =(\alpha-1) \Gamma \left( \alpha -1\right)
\end{equation}
Función de densidad de probabilidad Gamma con parámetros $\alpha$ y $\lambda$ cuando $x \geq 0$, $0$ de lo contrario:
\begin{equation}
    f(x)=\frac{\lambda e^{-\lambda x}(\lambda x)^{\alpha-1}}{\Gamma(\alpha)}
\end{equation}
La media de una variable aleatoria con distribución Gamma es:
\begin{equation}
    E(X) = \frac{\alpha}{\lambda}
\end{equation}
\subsection{Procedimientos}
La varianza se calcula con $E(X^{2})-E(X)^{2}$, como $E(X)$ ya es conocido, solo nos resta calcular $E(X^{2})$, la cual viene dada por:
\begin{equation}
    E(X^{2})=\int\limits_0^\infty {x^{2}f(x) dx}
\end{equation}
\begin{equation}
     =\int\limits_0^\infty {x^{2}\frac{\lambda e^{-\lambda x}(\lambda x)^{\alpha-1}}{\Gamma(\alpha)} dx}
\end{equation}
\begin{equation}
     =\frac{\lambda^{\alpha}}{\Gamma(\alpha)}\int\limits_0^\infty {x^{\alpha+1}e^{-\lambda x} dx}
\end{equation}
Para resolver la integral anterior se realiza la sustitución: $v=\lambda x$ y $dv=\lambda dx$
\begin{equation}
     =\frac{\lambda^{\alpha}}{\Gamma(\alpha)}\int\limits_0^\infty {\frac{v^{\alpha+1}}{\lambda^{\alpha+1}}e^{-v} \frac{dv}{\lambda}}
\end{equation}
\begin{equation}
        =\frac{\lambda^{\alpha}}{\lambda^{\alpha+2} \Gamma(\alpha)}\int\limits_0^\infty {v^{\alpha+1}e^{-v}} dv
\end{equation}
Sustituyendo la integral por la función Gamma antes mencionada:
\begin{equation}
    =\frac{\Gamma(\alpha+2)}{\lambda^{2}\Gamma(\alpha)}
\end{equation}
Haciendo uso de la propiedad de la función Gamma mencionada en los antecedentes de este ejercicio:
\begin{equation}
    \Gamma(\alpha+2)=(\alpha+1)\Gamma(\alpha+1)=(\alpha+1)\alpha\Gamma(\alpha)
\end{equation}
Por lo tanto:
\begin{equation}
    E(X^{2})=\frac{(\alpha+1)\alpha}{\lambda^{2}}
\end{equation}
Tomando el valor de $E(X)$ dado en los antecedentes, se encuentra que:
\begin{equation}
    Var(X)=\frac{(\alpha+1)\alpha}{\lambda^{2}}-\frac{\alpha^{2}}{\lambda^{2}}
\end{equation}
\begin{equation}
    =\frac{\alpha^{2}+\alpha}{\lambda^{2}}-\frac{\alpha^{2}}{\lambda^{2}}
\end{equation}
\begin{equation}
    =\frac{\alpha}{\lambda^{2}}
\end{equation}


\newpage
\bibliographystyle{unsrt}
\bibliography{ref}
\end{document}
